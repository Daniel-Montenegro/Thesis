%#########################################################################
%ABSTRACT


\begin{abstracts}
\selectlanguage{spanish}

Los actuales desarrollos en los modelos de estructura del Universo a gran escala, observaciones y las \'ultimas simulaciones con un poder de resoluci\'on mayor, a su vez muestran un Universo distribuido en regiones (estructuras). Estas estructuras  se distribuyen de una manera compleja y en forma de filamentos. Dichos filamentos representan una sobre densidad de materia y son producto de la no linealidad del sistema. La estructura que conforman el universo se le denomina la red c\'osmica. \\

La estructura del Universo plantea una duda que es el problema que se pretende resolver. ¿ Es posible encontrar una relaci\'on de las estructuras del Universo con las galaxias, o para ser m\'as específico con el spin de la misma?  Este trabajo nace la categorizaci\'on del mismo Universo, la particularidad de ciertas regiones da indicios de la relaci\'on entre entorno  galaxia; es por esto que se quiere conocer si la "mejor" relaci\'on es por parte del spin de las galaxias.

La presencia(existencia) de las regiones(estructuras) son resultado de la evoluci\'on del universo. \\

En estas regiones de sobre densidad se agrupan gran cantidad de materia que est\'a constituida por c\'umulos o agrupaciones de galaxias. Como su nombre lo dice, los c\'umulos o agrupaciones de galaxias son agrupaciones de galaxias debido a la interacci\'on gravitacional entre los cuerpos. Además, se conoce  la existencia de los agujeros negros al interior de las galaxias, el cual tiene asociado un spin. 

La motivaci\'on de este trabajo es poder encontrar una relaci\'on entre  \\


......



\end{abstracts}


%#########################################################################
